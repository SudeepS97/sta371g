\documentclass{beamer}\usepackage[]{graphicx}\usepackage[]{color}
%% maxwidth is the original width if it is less than linewidth
%% otherwise use linewidth (to make sure the graphics do not exceed the margin)
\makeatletter
\def\maxwidth{ %
  \ifdim\Gin@nat@width>\linewidth
    \linewidth
  \else
    \Gin@nat@width
  \fi
}
\makeatother

\definecolor{fgcolor}{rgb}{1, 0.894, 0.769}
\newcommand{\hlnum}[1]{\textcolor[rgb]{0.824,0.412,0.118}{#1}}%
\newcommand{\hlstr}[1]{\textcolor[rgb]{1,0.894,0.71}{#1}}%
\newcommand{\hlcom}[1]{\textcolor[rgb]{0.824,0.706,0.549}{#1}}%
\newcommand{\hlopt}[1]{\textcolor[rgb]{1,0.894,0.769}{#1}}%
\newcommand{\hlstd}[1]{\textcolor[rgb]{1,0.894,0.769}{#1}}%
\newcommand{\hlkwa}[1]{\textcolor[rgb]{0.941,0.902,0.549}{#1}}%
\newcommand{\hlkwb}[1]{\textcolor[rgb]{0.804,0.776,0.451}{#1}}%
\newcommand{\hlkwc}[1]{\textcolor[rgb]{0.78,0.941,0.545}{#1}}%
\newcommand{\hlkwd}[1]{\textcolor[rgb]{1,0.78,0.769}{#1}}%
\let\hlipl\hlkwb

\usepackage{framed}
\makeatletter
\newenvironment{kframe}{%
 \def\at@end@of@kframe{}%
 \ifinner\ifhmode%
  \def\at@end@of@kframe{\end{minipage}}%
  \begin{minipage}{\columnwidth}%
 \fi\fi%
 \def\FrameCommand##1{\hskip\@totalleftmargin \hskip-\fboxsep
 \colorbox{shadecolor}{##1}\hskip-\fboxsep
     % There is no \\@totalrightmargin, so:
     \hskip-\linewidth \hskip-\@totalleftmargin \hskip\columnwidth}%
 \MakeFramed {\advance\hsize-\width
   \@totalleftmargin\z@ \linewidth\hsize
   \@setminipage}}%
 {\par\unskip\endMakeFramed%
 \at@end@of@kframe}
\makeatother

\definecolor{shadecolor}{rgb}{.97, .97, .97}
\definecolor{messagecolor}{rgb}{0, 0, 0}
\definecolor{warningcolor}{rgb}{1, 0, 1}
\definecolor{errorcolor}{rgb}{1, 0, 0}
\newenvironment{knitrout}{}{} % an empty environment to be redefined in TeX

\usepackage{alltt}
\usepackage{preview}
\usepackage{../371g-slides}
\title{Model evaluation: training and test sets}
\subtitle{Lecture 21}
\author{STA 371G}
\IfFileExists{upquote.sty}{\usepackage{upquote}}{}
\begin{document}
  

  

  \frame{\maketitle}

  % Show outline at beginning of each section
  \AtBeginSection[]{
    \begin{frame}<beamer>
      \tableofcontents[currentsection]
    \end{frame}
  }

  %%%%%%% Slides start here %%%%%%%

  \begin{darkframes}
    \begin{frame}{Overfitting}
      \begin{itemize}
        \item A related problem to $p$-hacking is the issue of \alert{overfitting}: creating a model that fits your \emph{sample} very well but does not generalize well to the larger \emph{population}.
        \item In other words, an overfit model is one where the $R^2$ (for linear regression) or prediction accuracy (for logistic regression) is high, but the model will not work as well as expected when given new data.
        \item Let's consider a simple problem of predicting $Y$ from one $X$, and fitting a polynomial curve to the data.
      \end{itemize}
    \end{frame}

    \begin{frame}
\begin{knitrout}
\definecolor{shadecolor}{rgb}{0.137, 0.137, 0.137}
\input{/tmp/figures/unnamed-chunk-2-1.tex}

\end{knitrout}

      \pause
      \vspace{-1cm}
      As we increase the degree (the highest power of $x$) of the polynomial, the fit improves, since the curve can zig and zag to capture more of the idiosyncrasies of the sample.
    \end{frame}

    \begin{frame}
      \fullpagepicture{polynomials}
    \end{frame}

    \begin{frame}{Degree 1 polynomial}
\begin{knitrout}
\definecolor{shadecolor}{rgb}{0.137, 0.137, 0.137}
\input{/tmp/figures/unnamed-chunk-3-1.tex}

\end{knitrout}
      \vspace{-1cm} $\text{average absolute prediction error} = 0.3438$
    \end{frame}

    \begin{frame}{Degree 2 polynomial}
\begin{knitrout}
\definecolor{shadecolor}{rgb}{0.137, 0.137, 0.137}
\input{/tmp/figures/unnamed-chunk-4-1.tex}

\end{knitrout}
      \vspace{-1cm} $\text{average absolute prediction error} = 0.3097$
    \end{frame}

    \begin{frame}{Degree 3 polynomial}
\begin{knitrout}
\definecolor{shadecolor}{rgb}{0.137, 0.137, 0.137}
\input{/tmp/figures/unnamed-chunk-5-1.tex}

\end{knitrout}
      \vspace{-1cm} $\text{average absolute prediction error} = 0.1696$
    \end{frame}

    \begin{frame}{Degree 10 polynomial}
\begin{knitrout}
\definecolor{shadecolor}{rgb}{0.137, 0.137, 0.137}
\input{/tmp/figures/unnamed-chunk-6-1.tex}

\end{knitrout}
      \vspace{-1cm} $\text{average absolute prediction error} = 0.1565$
    \end{frame}

    \begin{frame}{Degree 20 polynomial}
\begin{knitrout}
\definecolor{shadecolor}{rgb}{0.137, 0.137, 0.137}
\input{/tmp/figures/unnamed-chunk-7-1.tex}

\end{knitrout}
      \vspace{-1cm} $\text{average absolute prediction error} = 0.156$
    \end{frame}

    \begin{frame}{Degree 30 polynomial}
\begin{knitrout}
\definecolor{shadecolor}{rgb}{0.137, 0.137, 0.137}
\input{/tmp/figures/unnamed-chunk-8-1.tex}

\end{knitrout}
      \vspace{-1cm} $\text{average absolute prediction error} = 0.1559$
    \end{frame}

    \begin{frame}
      As we fit increasingly complex polynomials, the average prediction error decreases:

\begin{knitrout}
\definecolor{shadecolor}{rgb}{0.137, 0.137, 0.137}
\input{/tmp/figures/unnamed-chunk-9-1.tex}

\end{knitrout}
    \end{frame}

    \begin{frame}{Parsimony}
      \begin{itemize}
        \item We've talked a lot about the need to select models that are \alert{parsimonious}, i.e., those that strike a good balance between fitting well and being simple.
        \item Besides being easier to communicate, there's another reason to prefer simpler models---they tend to generalize better to new data.
        \item Let's see how these models perform on a new sample from the same population!
      \end{itemize}
    \end{frame}

    \begin{frame}{Degree 1 polynomial}
\begin{knitrout}
\definecolor{shadecolor}{rgb}{0.137, 0.137, 0.137}
\input{/tmp/figures/unnamed-chunk-10-1.tex}

\end{knitrout}
      \vspace{-1cm} $\text{average absolute prediction error} = 0.3339$
      (was $0.3438$ in the original sample)
    \end{frame}

    \begin{frame}{Degree 2 polynomial}
\begin{knitrout}
\definecolor{shadecolor}{rgb}{0.137, 0.137, 0.137}
\input{/tmp/figures/unnamed-chunk-11-1.tex}

\end{knitrout}
      \vspace{-1cm} $\text{average absolute prediction error} = 0.3688$
      (was $0.3097$ in the original sample)
    \end{frame}

    \begin{frame}{Degree 3 polynomial}
\begin{knitrout}
\definecolor{shadecolor}{rgb}{0.137, 0.137, 0.137}
\input{/tmp/figures/unnamed-chunk-12-1.tex}

\end{knitrout}
      \vspace{-1cm} $\text{average absolute prediction error} = 0.2718$
      (was $0.1696$ in the original sample)
    \end{frame}

    \begin{frame}{Degree 10 polynomial}
\begin{knitrout}
\definecolor{shadecolor}{rgb}{0.137, 0.137, 0.137}
\input{/tmp/figures/unnamed-chunk-13-1.tex}

\end{knitrout}
      \vspace{-1cm} $\text{average absolute prediction error} = 0.2927$
      (was $0.1565$ in the original sample)
    \end{frame}

    \begin{frame}{Degree 20 polynomial}
\begin{knitrout}
\definecolor{shadecolor}{rgb}{0.137, 0.137, 0.137}
\input{/tmp/figures/unnamed-chunk-14-1.tex}

\end{knitrout}
      \vspace{-1cm} $\text{average absolute prediction error} = 0.2927$
      (was $0.156$ in the original sample)
    \end{frame}

    \begin{frame}{Degree 30 polynomial}
\begin{knitrout}
\definecolor{shadecolor}{rgb}{0.137, 0.137, 0.137}
\input{/tmp/figures/unnamed-chunk-15-1.tex}

\end{knitrout}
      \vspace{-1cm} $\text{average absolute prediction error} = 0.2927$
      (was $0.1559$ in the original sample)
    \end{frame}

    \begin{frame}
      The increasingly complex polynomials do not perform as well on the new data as the simple polynomials, because they had \alert{overfit} the idiosyncrasies of the original data.

\begin{knitrout}
\definecolor{shadecolor}{rgb}{0.137, 0.137, 0.137}
\input{/tmp/figures/unnamed-chunk-16-1.tex}

\end{knitrout}
    \end{frame}

    \begin{frame}{Overfitting in multiple regression}
      \begin{itemize}[<+->]
        \item A similar phenomenon occurs when you have many variables to choose from when building a regression model.
        \item By including many variables in your model, you can get $R^2$ to increase (or prediction error to decrease) on the data used to build the model.
        \item But the more complex models are more likely to be \alert{overfitting} the data, and won't generalize as well as simple models.
        \item In general, a model's performance on the data used to build the model will usually be stronger than its performance on new data.
        \item We can use the concept of training and test sets to address both $p$-hacking and overfitting.
      \end{itemize}
    \end{frame}

    \begin{frame}{Review of training and test sets}
      \begin{center}
        \tikzstyle{block} = [rectangle, draw, fill=darkgray,
          text centered, minimum height=2em]
        \tikzstyle{line} = [draw, -latex']

        \begin{tikzpicture}[auto]
          \node [block, text width = 10cm] (all) {Original data set};
          \node [block, below of=all, text width = 7cm, xshift=-1.5cm] (training) {Training set};
          \node [block, right of=training, node distance = 5cm, text width = 3cm] (test) {Test set};
        \end{tikzpicture}
      \end{center}

      \begin{itemize}
        \item Split the data into a \alert{training set} and a \alert{test set} (a typical split is 70\% training set / 30\% test set).
        \item We use the training set to build the model, and then evaluate the quality of the model on how well it predicts $Y$ in the test set.
      \end{itemize}
    \end{frame}

    \begin{frame}{The housing data set}
      Let's look at three different models to predict housing prices, with three different sets of predictors:

      \begin{itemize}
        \item Model 1: living area only (\texttt{Living.Area})
        \item Model 2: living area and whether the house is on the water (\texttt{Living.Area} and \texttt{Waterfront})
        \item Model 3: living area, whether the house is on the water, and whether the house uses electric, gas, or oil (\texttt{Living.Area}, \texttt{Waterfront}, and \texttt{Fuel.Type})
      \end{itemize}
    \end{frame}

    \begin{frame}[fragile]
      Let's split our data into a training and test set with a rough 70/30 split. ($N=1728$, and 30\% of 1728 is approximately 518.)

      \fontsm
\begin{knitrout}
\definecolor{shadecolor}{rgb}{0.137, 0.137, 0.137}\begin{kframe}
\begin{alltt}
\hlcom{# Select which row numbers will correspond to test cases}
\hlstd{test.cases} \hlkwb{<-} \hlkwd{sample}\hlstd{(}\hlnum{1}\hlopt{:}\hlnum{1728}\hlstd{,} \hlnum{518}\hlstd{)}

\hlcom{# Build the test set from those row numbers}
\hlstd{test.set} \hlkwb{<-} \hlstd{houses[test.cases,]}

\hlcom{# Remaining row numbers will correspond to training cases}
\hlstd{training.cases} \hlkwb{<-} \hlkwd{setdiff}\hlstd{(}\hlnum{1}\hlopt{:}\hlnum{1728}\hlstd{, test.cases)}

\hlcom{# Build the training set from those row numbers}
\hlstd{training.set} \hlkwb{<-} \hlstd{houses[training.cases,]}
\end{alltt}
\end{kframe}
\end{knitrout}
    \end{frame}

    \begin{frame}{Model-building strategy}
      \begin{itemize}[<+->]
        \item At first, the temptation is to build models on the training set, test using the test set, and repeat.
        \item But this would lead to the same overfitting and $p$-hacking issues as before!
        \item Instead, set aside the test set and don't look at it until you have a final model: what you think is the most parsimonious model.
        \item Evaluate model performance in the test set as a way to get a fair measurement of how well your model will perform on new data.
      \end{itemize}
    \end{frame}

    \begin{frame}[fragile]{Evaluating training and test error}
      \fontsm
      Let's start with a simple model that uses living area only:
\begin{knitrout}
\definecolor{shadecolor}{rgb}{0.137, 0.137, 0.137}\begin{kframe}
\begin{alltt}
\hlstd{model} \hlkwb{<-} \hlkwd{lm}\hlstd{(Price} \hlopt{~} \hlstd{Living.Area,} \hlkwc{data}\hlstd{=training.set)}
\end{alltt}
\end{kframe}
\end{knitrout}
      The training set average error is:
\begin{knitrout}
\definecolor{shadecolor}{rgb}{0.137, 0.137, 0.137}\begin{kframe}
\begin{alltt}
\hlkwd{mean}\hlstd{(}\hlkwd{abs}\hlstd{(}\hlkwd{resid}\hlstd{(model)))}
\end{alltt}
\begin{verbatim}
[1] 46371.65
\end{verbatim}
\end{kframe}
\end{knitrout}
      The test set average error comes from manually computing the prediction error for each case in the test set:
\begin{knitrout}
\definecolor{shadecolor}{rgb}{0.137, 0.137, 0.137}\begin{kframe}
\begin{alltt}
\hlstd{price.hat} \hlkwb{<-} \hlkwd{predict}\hlstd{(model, test.set)}
\hlkwd{mean}\hlstd{(}\hlkwd{abs}\hlstd{(test.set}\hlopt{$}\hlstd{Price} \hlopt{-} \hlstd{price.hat))}
\end{alltt}
\begin{verbatim}
[1] 50596.29
\end{verbatim}
\end{kframe}
\end{knitrout}
    \end{frame}

    \begin{frame}[fragile]{Evaluating training and test error}
      \begin{itemize}[<+->]
        \item Similarly, we can compare the $R^2$ from the training set to the $R^2$ that we would get by predicting prices for cases in the test set.
        \item Recall that $R^2 = \text{cor}(Y,\hat Y)^2$; we can simulate what $R^2$ would be in the test set by calculating this in the test set:
\begin{knitrout}
\definecolor{shadecolor}{rgb}{0.137, 0.137, 0.137}\begin{kframe}
\begin{alltt}
\hlkwd{cor}\hlstd{(test.set}\hlopt{$}\hlstd{Price, price.hat)}\hlopt{^}\hlnum{2}
\end{alltt}
\begin{verbatim}
[1] 0.4817818
\end{verbatim}
\end{kframe}
\end{knitrout}

        \item Compare this to what $R^2$ is in the training set:
\begin{knitrout}
\definecolor{shadecolor}{rgb}{0.137, 0.137, 0.137}\begin{kframe}
\begin{alltt}
\hlkwd{summary}\hlstd{(model)}\hlopt{$}\hlstd{r.squared}
\end{alltt}
\begin{verbatim}
[1] 0.5200801
\end{verbatim}
\end{kframe}
\end{knitrout}
      \end{itemize}
    \end{frame}

    \begin{frame}{Using training and test sets in logistic regression}
      \begin{itemize}
        \item The logic of training and test sets is identical for logistic regression.
        \item If we build a logistic model to predict something, using a test set for evaluation will give us a more realistic estimate of prediction accuracy on new data.
        \item Let's try this out on a data set of passengers from the \emph{Titanic}.
      \end{itemize}
    \end{frame}

    \begin{frame}[fragile]{The data set}
      The \emph{Titanic} data set has data on 756 passengers on the \emph{Titanic}; we'll predict the categorical variable survival from age of the passenger.  Let's segment the data into training and test sets, as before:

      \fontsm
\begin{knitrout}
\definecolor{shadecolor}{rgb}{0.137, 0.137, 0.137}\begin{kframe}
\begin{alltt}
\hlcom{# Select which row numbers will correspond to test cases}
\hlstd{test.cases} \hlkwb{<-} \hlkwd{sample}\hlstd{(}\hlnum{1}\hlopt{:}\hlnum{756}\hlstd{,} \hlnum{227}\hlstd{)}

\hlcom{# Build the test set from those row numbers}
\hlstd{test.set} \hlkwb{<-} \hlstd{titanic[test.cases,]}

\hlcom{# Remaining row numbers will correspond to training cases}
\hlstd{training.cases} \hlkwb{<-} \hlkwd{setdiff}\hlstd{(}\hlnum{1}\hlopt{:}\hlnum{756}\hlstd{, test.cases)}

\hlcom{# Build the training set from those row numbers}
\hlstd{training.set} \hlkwb{<-} \hlstd{titanic[training.cases,]}
\end{alltt}
\end{kframe}
\end{knitrout}
    \end{frame}

    \begin{frame}[fragile]
      \fontsm
      Let's build the model:
\begin{knitrout}
\definecolor{shadecolor}{rgb}{0.137, 0.137, 0.137}\begin{kframe}
\begin{alltt}
\hlstd{model} \hlkwb{<-} \hlkwd{glm}\hlstd{(Survived} \hlopt{~} \hlstd{Age,} \hlkwc{data}\hlstd{=training.set,} \hlkwc{family}\hlstd{=binomial)}
\end{alltt}
\end{kframe}
\end{knitrout}

      \pause

      Then we can evaluate prediction accuracy as usual:
\begin{knitrout}
\definecolor{shadecolor}{rgb}{0.137, 0.137, 0.137}\begin{kframe}
\begin{alltt}
\hlstd{predicted} \hlkwb{<-} \hlstd{(}\hlkwd{predict}\hlstd{(model,} \hlkwc{type}\hlstd{=}\hlstr{"response"}\hlstd{)} \hlopt{>=} \hlnum{0.5}\hlstd{)}
\hlstd{actual} \hlkwb{<-} \hlstd{(training.set}\hlopt{$}\hlstd{Survived} \hlopt{==} \hlnum{1}\hlstd{)}
\hlkwd{sum}\hlstd{(predicted} \hlopt{==} \hlstd{actual)} \hlopt{/} \hlkwd{nrow}\hlstd{(training.set)}
\end{alltt}
\begin{verbatim}
[1] 0.5879017
\end{verbatim}
\end{kframe}
\end{knitrout}

      \pause

      To evaluate prediction accuracy on the training set, we use the same model, but apply it to the test set:
\begin{knitrout}
\definecolor{shadecolor}{rgb}{0.137, 0.137, 0.137}\begin{kframe}
\begin{alltt}
\hlstd{predicted} \hlkwb{<-} \hlstd{(}\hlkwd{predict}\hlstd{(model, test.set,} \hlkwc{type}\hlstd{=}\hlstr{"response"}\hlstd{)} \hlopt{>=} \hlnum{0.5}\hlstd{)}
\hlstd{actual} \hlkwb{<-} \hlstd{(test.set}\hlopt{$}\hlstd{Survived} \hlopt{==} \hlnum{1}\hlstd{)}
\hlkwd{sum}\hlstd{(predicted} \hlopt{==} \hlstd{actual)} \hlopt{/} \hlkwd{nrow}\hlstd{(test.set)}
\end{alltt}
\begin{verbatim}
[1] 0.5814978
\end{verbatim}
\end{kframe}
\end{knitrout}
    \end{frame}

    \begin{frame}{Cross-validation}
      \begin{center}
        \tikzstyle{block} = [rectangle, draw, fill=darkgray,
          text centered, minimum height=2em]
        \tikzstyle{line} = [draw, -latex']

        \begin{tikzpicture}[auto]
          \node [block, text width = 9.75cm] (all) {Original data set};
          \node [block, below of=all, text width = 1.75cm, xshift=-4cm] (fold1) {Fold 1};
          \node [block, right of=fold1, node distance = 2cm, text width = 1.75cm] (fold2) {Fold 2};
          \node [block, right of=fold2, node distance = 2cm, text width = 1.75cm] (fold3) {Fold 3};
          \node [block, right of=fold3, node distance = 2cm, text width = 1.75cm] (fold4) {Fold 4};
          \node [block, right of=fold4, node distance = 2cm, text width = 1.75cm] (fold5) {Fold 5};
        \end{tikzpicture}
      \end{center}

      \begin{itemize}
        \item Split the data into $k$ ``folds'' (here $k=5$).
        \item For each fold, use that fold as a test set and the other folds together as a training set.
        \item Average the prediction accuracy across all $k$ folds as your best estimate of prediction accuracy.
        \item Cross-validation reduces the impact of the random chance from your selection of training and test sets.
      \end{itemize}
    \end{frame}
  \end{darkframes}
\end{document}

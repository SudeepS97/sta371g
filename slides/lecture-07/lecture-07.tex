\documentclass{beamer}\usepackage[]{graphicx}\usepackage[]{color}
%% maxwidth is the original width if it is less than linewidth
%% otherwise use linewidth (to make sure the graphics do not exceed the margin)
\makeatletter
\def\maxwidth{ %
  \ifdim\Gin@nat@width>\linewidth
    \linewidth
  \else
    \Gin@nat@width
  \fi
}
\makeatother

\definecolor{fgcolor}{rgb}{1, 0.894, 0.769}
\newcommand{\hlnum}[1]{\textcolor[rgb]{0.824,0.412,0.118}{#1}}%
\newcommand{\hlstr}[1]{\textcolor[rgb]{1,0.894,0.71}{#1}}%
\newcommand{\hlcom}[1]{\textcolor[rgb]{0.824,0.706,0.549}{#1}}%
\newcommand{\hlopt}[1]{\textcolor[rgb]{1,0.894,0.769}{#1}}%
\newcommand{\hlstd}[1]{\textcolor[rgb]{1,0.894,0.769}{#1}}%
\newcommand{\hlkwa}[1]{\textcolor[rgb]{0.941,0.902,0.549}{#1}}%
\newcommand{\hlkwb}[1]{\textcolor[rgb]{0.804,0.776,0.451}{#1}}%
\newcommand{\hlkwc}[1]{\textcolor[rgb]{0.78,0.941,0.545}{#1}}%
\newcommand{\hlkwd}[1]{\textcolor[rgb]{1,0.78,0.769}{#1}}%
\let\hlipl\hlkwb

\usepackage{framed}
\makeatletter
\newenvironment{kframe}{%
 \def\at@end@of@kframe{}%
 \ifinner\ifhmode%
  \def\at@end@of@kframe{\end{minipage}}%
  \begin{minipage}{\columnwidth}%
 \fi\fi%
 \def\FrameCommand##1{\hskip\@totalleftmargin \hskip-\fboxsep
 \colorbox{shadecolor}{##1}\hskip-\fboxsep
     % There is no \\@totalrightmargin, so:
     \hskip-\linewidth \hskip-\@totalleftmargin \hskip\columnwidth}%
 \MakeFramed {\advance\hsize-\width
   \@totalleftmargin\z@ \linewidth\hsize
   \@setminipage}}%
 {\par\unskip\endMakeFramed%
 \at@end@of@kframe}
\makeatother

\definecolor{shadecolor}{rgb}{.97, .97, .97}
\definecolor{messagecolor}{rgb}{0, 0, 0}
\definecolor{warningcolor}{rgb}{1, 0, 1}
\definecolor{errorcolor}{rgb}{1, 0, 0}
\newenvironment{knitrout}{}{} % an empty environment to be redefined in TeX

\usepackage{alltt}
\usepackage{../371g-slides}
\title{Multiple regression 1}
\subtitle{Lecture 7}
\author{STA 371G}
\IfFileExists{upquote.sty}{\usepackage{upquote}}{}
\begin{document}
  
  

  \frame{\maketitle}

  % Show outline at beginning of each section
  \AtBeginSection[]{
    \begin{frame}<beamer>
      \tableofcontents[currentsection]
    \end{frame}
  }

  %%%%%%% Slides start here %%%%%%%

  \begin{darkframes}
    \begin{frame}{Why do some colleges have higher graduation rates than others?}
      \begin{itemize}[<+->]
        \item What factors do you think impact the graduation rate of a college?
        \item It seems like there is no \emph{one} factor that dominates---it is probably true that to make a good prediction we need to put a lot of variables together, so simple regression is likely not sufficient.
        \item \alert{Multiple regression} allows us to build on simple regression by predicting one $Y$ variable using multiple $X$ variables.
      \end{itemize}
    \end{frame}

    \begin{frame}{The colleges data set}
      Today's data set is a sample of 1302 colleges with various factors about the colleges, including SAT scores, student/faculty ratios, tuition rates, acceptance rates, etc.
    \end{frame}

    \begin{frame}[fragile]

      \begin{center}
        SAT scores and (in-state) tuition were the two best single predictors, with $R^2$ values of 0.353 and 0.325, respectively. Can we combine these together and get an $R^2$ that is better than either predictor would produce on its own?
      \end{center}
    \end{frame}

    \begin{frame}{Using multiple predictors to predict graduation rate}
      The simple regression models were:
      \[
        Y_i = \beta_0 + \beta_1 (\text{SAT}) + \epsilon_i
      \]
      and
      \[
        Y_i = \beta_0 + \beta_1 (\text{tuition}) + \epsilon_i.
      \]
      The multiple regression model is
      \[
        Y_i = \beta_0 + \beta_1 (\text{tuition}) + \beta_2 (\text{SAT}) + \epsilon_i.
      \]
    \end{frame}

    \begin{frame}[fragile]
      \fontsize{8}{8}
\begin{knitrout}
\definecolor{shadecolor}{rgb}{0.137, 0.137, 0.137}\begin{kframe}
\begin{alltt}
\hlstd{> }\hlstd{model} \hlkwb{<-} \hlkwd{lm}\hlstd{(Graduation.rate} \hlopt{~} \hlstd{Average.combined.SAT} \hlopt{+} \hlstd{In.state.tuition,} \hlkwc{data}\hlstd{=my.sample)}
\hlstd{> }\hlkwd{summary}\hlstd{(model)}
\end{alltt}
\begin{verbatim}

Call:
lm(formula = Graduation.rate ~ Average.combined.SAT + In.state.tuition, 
    data = my.sample)

Residuals:
   Min     1Q Median     3Q    Max 
-45.53  -9.18   0.05   8.70  43.66 

Coefficients:
                      Estimate Std. Error t value Pr(>|t|)    
(Intercept)          -8.324646   4.370828    -1.9    0.057 .  
Average.combined.SAT  0.061122   0.004888    12.5   <2e-16 ***
In.state.tuition      0.001249   0.000111    11.2   <2e-16 ***
---
Signif. codes:  0 '***' 0.001 '**' 0.01 '*' 0.05 '.' 0.1 ' ' 1

Residual standard error: 13.7 on 709 degrees of freedom
  (19 observations deleted due to missingness)
Multiple R-squared:  0.447,	Adjusted R-squared:  0.445 
F-statistic:  286 on 2 and 709 DF,  p-value: <2e-16
\end{verbatim}
\end{kframe}
\end{knitrout}
    \end{frame}

    \begin{frame}
      The multiple regression prediction equation is:

      \[
        \widehat{\text{Graduation rate}} =
          -8.3246 +
          0.0611(\text{SAT})
          + 0.0012(\text{tuition})
      \]
      \pause
      We can use this to make predictions like we would for a simple regression!
    \end{frame}

    \begin{frame}[fragile]
\begin{knitrout}
\definecolor{shadecolor}{rgb}{0.137, 0.137, 0.137}
\input{/tmp/figures/unnamed-chunk-5-1.tex}

\end{knitrout}
    \end{frame}

    \begin{frame}[fragile]
\begin{knitrout}
\definecolor{shadecolor}{rgb}{0.137, 0.137, 0.137}
\input{/tmp/figures/unnamed-chunk-6-1.tex}

\end{knitrout}
    \end{frame}

    \begin{frame}{Interpreting the coefficients: intercept}
      Let's interpret the intercept coefficient of $-8.3246$:
      \begin{itemize}[<+->]
        \item The predicted graduation rate when the average SAT score is 0 and the in-state tuition is \$0 is $-8.3246$.
        \item This is not a meaningful number on its own in this case, since there will never be a school with those particular predictor values! (The intercept might be interpretable for other models.)
      \end{itemize}
    \end{frame}

    \begin{frame}{Interpreting the coefficients: SAT}
      Let's interpret the SAT coefficient of 0.0611:
      \begin{itemize}[<+->]
        \item \alert{Holding tuition constant}, each additional SAT score point  increases our predicted graduation rate by 0.0611 percentage points.
        \item \alert{Among colleges that have the same tuition}, an increase in SAT of 1 point would result in a predicted graduation rate that is 0.0611 percentage points higher.
        \item \alert{If we compared two colleges that have the same tuition but differ in average SAT scores by 1 point}, the college with the higher SAT score would be predicted to have a graduation rate that is 0.0611 percentage points higher.
      \end{itemize}
    \end{frame}

    \begin{frame}{Interpreting the coefficients: tuition}
      Let's interpret the tuition coefficient of 0.0012:
      \begin{itemize}[<+->]
        \item \alert{Holding SAT constant}, each additional dollar of in-state tuition increases our predicted graduation rate by 0.0012 percentage points.
        \item \alert{Among colleges that have the same average SAT scores}, an increase in tuition of \$1 would result in a predicted graduation rate that is 0.0012 percentage points higher.
        \item \alert{If we compared two colleges that have the same average SAT scores but differ in their tuition by \$1}, the college with the higher tuition would be predicted to have a graduation rate that is 0.0012 percentage points higher.
      \end{itemize}
    \end{frame}

    \begin{frame}{What's the difference?!}
      \begin{itemize}[<+->]
        \item The difference between ``the predicted effect of a 1-point increase in SAT score'' and ``the predicted effect of a 1-point increase in SAT score, holding tuition constant'' really are \alert{two different things}.
        \item The relationship between $X_1$ and $Y$ may change when we \alert{control for} (i.e., add to the model) another predictor $X_2$.
      \end{itemize}
    \end{frame}

    \begin{frame}{$R^2$}
      \begin{itemize}[<+->]
        \item $R^2$ has a similar meaning as in simple regression: how much of the variation in the response variable ($Y$) are explained by the predictor variables ($X$'s) together?
        \item Another way to think about $R^2$ is that \[ R^2 = \frac{\text{Var}(\hat Y)}{\text{Var}(Y)}, \] i.e., it represents how much variance in $Y$ the model predicts.
        \item $R^2$ always increases when you add more variables, \alert{even if you add variables that have no real relationship with $Y$}.
      \end{itemize}
    \end{frame}

    \begin{frame}[fragile]
      
      \fontsize{8}{8}
\begin{knitrout}
\definecolor{shadecolor}{rgb}{0.137, 0.137, 0.137}\begin{kframe}
\begin{alltt}
\hlstd{> }\hlstd{model1} \hlkwb{<-} \hlkwd{lm}\hlstd{(Graduation.rate} \hlopt{~} \hlstd{Average.combined.SAT} \hlopt{+} \hlstd{In.state.tuition,}
\hlstd{+ }             \hlkwc{data}\hlstd{=my.sample)}
\hlstd{> }\hlkwd{summary}\hlstd{(model1)}
\end{alltt}
\begin{verbatim}

Call:
lm(formula = Graduation.rate ~ Average.combined.SAT + In.state.tuition, 
    data = my.sample)

Residuals:
      Min        1Q    Median        3Q       Max 
-45.52572  -9.18156   0.05085   8.70420  43.66097 

Coefficients:
                         Estimate   Std. Error  t value Pr(>|t|)    
(Intercept)          -8.324645625  4.370827909 -1.90459 0.057238 .  
Average.combined.SAT  0.061122082  0.004887825 12.50496  < 2e-16 ***
In.state.tuition      0.001248638  0.000111119 11.23692  < 2e-16 ***
---
Signif. codes:  0 '***' 0.001 '**' 0.01 '*' 0.05 '.' 0.1 ' ' 1

Residual standard error: 13.7466 on 709 degrees of freedom
  (19 observations deleted due to missingness)
Multiple R-squared:  0.44686,	Adjusted R-squared:   0.4453 
F-statistic: 286.387 on 2 and 709 DF,  p-value: < 2.22e-16
\end{verbatim}
\end{kframe}
\end{knitrout}
    \end{frame}

    \begin{frame}[fragile]
      \fontsize{8}{8}
\begin{knitrout}
\definecolor{shadecolor}{rgb}{0.137, 0.137, 0.137}\begin{kframe}
\begin{alltt}
\hlstd{> }\hlstd{Random.numbers} \hlkwb{<-} \hlkwd{rnorm}\hlstd{(}\hlkwd{nrow}\hlstd{(my.sample))}
\hlstd{> }\hlstd{model2} \hlkwb{<-} \hlkwd{lm}\hlstd{(Graduation.rate} \hlopt{~} \hlstd{Average.combined.SAT} \hlopt{+} \hlstd{In.state.tuition}
\hlstd{+ }               \hlopt{+} \hlstd{Random.numbers,} \hlkwc{data}\hlstd{=my.sample)}
\hlstd{> }\hlkwd{summary}\hlstd{(model2)}
\end{alltt}
\begin{verbatim}

Call:
lm(formula = Graduation.rate ~ Average.combined.SAT + In.state.tuition + 
    Random.numbers, data = my.sample)

Residuals:
      Min        1Q    Median        3Q       Max 
-45.59477  -9.13473   0.06836   8.75583  43.74968 

Coefficients:
                         Estimate   Std. Error  t value Pr(>|t|)    
(Intercept)          -8.433559857  4.378188630 -1.92627 0.054471 .  
Average.combined.SAT  0.061244215  0.004896088 12.50881  < 2e-16 ***
In.state.tuition      0.001248531  0.000111177 11.23012  < 2e-16 ***
Random.numbers        0.277098090  0.537299499  0.51572 0.606208    
---
Signif. codes:  0 '***' 0.001 '**' 0.01 '*' 0.05 '.' 0.1 ' ' 1

Residual standard error: 13.7537 on 708 degrees of freedom
  (19 observations deleted due to missingness)
Multiple R-squared:  0.447068,	Adjusted R-squared:  0.444725 
F-statistic: 190.816 on 3 and 708 DF,  p-value: < 2.22e-16
\end{verbatim}
\end{kframe}
\end{knitrout}
    \end{frame}

    \begin{frame}[fragile]
      \fontsize{8}{8}
\begin{knitrout}
\definecolor{shadecolor}{rgb}{0.137, 0.137, 0.137}\begin{kframe}
\begin{alltt}
\hlstd{> }\hlstd{Random.numbers} \hlkwb{<-} \hlkwd{rnorm}\hlstd{(}\hlkwd{nrow}\hlstd{(my.sample))}
\hlstd{> }\hlstd{model2} \hlkwb{<-} \hlkwd{lm}\hlstd{(Graduation.rate} \hlopt{~} \hlstd{Average.combined.SAT} \hlopt{+} \hlstd{In.state.tuition}
\hlstd{+ }               \hlopt{+} \hlstd{Average.math.SAT,} \hlkwc{data}\hlstd{=my.sample)}
\hlstd{> }\hlkwd{summary}\hlstd{(model2)}
\end{alltt}
\begin{verbatim}

Call:
lm(formula = Graduation.rate ~ Average.combined.SAT + In.state.tuition + 
    Average.math.SAT, data = my.sample)

Residuals:
      Min        1Q    Median        3Q       Max 
-45.27189  -9.06503   0.03009   8.64981  43.89591 

Coefficients:
                         Estimate   Std. Error  t value Pr(>|t|)    
(Intercept)          -8.144252350  4.434188943 -1.83669 0.066675 .  
Average.combined.SAT  0.054229967  0.023519872  2.30571 0.021416 *  
In.state.tuition      0.001256312  0.000115918 10.83790  < 2e-16 ***
Average.math.SAT      0.012667133  0.041953872  0.30193 0.762794    
---
Signif. codes:  0 '***' 0.001 '**' 0.01 '*' 0.05 '.' 0.1 ' ' 1

Residual standard error: 13.7492 on 706 degrees of freedom
  (21 observations deleted due to missingness)
Multiple R-squared:  0.447693,	Adjusted R-squared:  0.445346 
F-statistic: 190.758 on 3 and 706 DF,  p-value: < 2.22e-16
\end{verbatim}
\end{kframe}
\end{knitrout}
    \end{frame}

    \begin{frame}{Adjusted $R^2$}
      \begin{itemize}[<+->]
        \item There are many, many possible models (think of how many combinations of predictors there are!) so we need some criterion to determine which model is best.
        \item $R^2$ is not good because adding even a variable of random numbers increases $R^2$.
        \item \alert{Adjusted $R^2$} makes an adjustment to $R^2$ by adding a penalty for each variable added (in this example, adjusted $R^2$ went down even though $R^2$ increased).
      \end{itemize}
    \end{frame}

    \begin{frame}[fragile]
      
      \fontsize{8}{8}
\begin{knitrout}
\definecolor{shadecolor}{rgb}{0.137, 0.137, 0.137}\begin{kframe}
\begin{alltt}
\hlstd{> }\hlstd{model1} \hlkwb{<-} \hlkwd{lm}\hlstd{(Graduation.rate} \hlopt{~} \hlstd{Average.combined.SAT} \hlopt{+} \hlstd{In.state.tuition,}
\hlstd{+ }             \hlkwc{data}\hlstd{=my.sample)}
\hlstd{> }\hlkwd{summary}\hlstd{(model1)}
\end{alltt}
\begin{verbatim}

Call:
lm(formula = Graduation.rate ~ Average.combined.SAT + In.state.tuition, 
    data = my.sample)

Residuals:
      Min        1Q    Median        3Q       Max 
-45.52572  -9.18156   0.05085   8.70420  43.66097 

Coefficients:
                         Estimate   Std. Error  t value Pr(>|t|)    
(Intercept)          -8.324645625  4.370827909 -1.90459 0.057238 .  
Average.combined.SAT  0.061122082  0.004887825 12.50496  < 2e-16 ***
In.state.tuition      0.001248638  0.000111119 11.23692  < 2e-16 ***
---
Signif. codes:  0 '***' 0.001 '**' 0.01 '*' 0.05 '.' 0.1 ' ' 1

Residual standard error: 13.7466 on 709 degrees of freedom
  (19 observations deleted due to missingness)
Multiple R-squared:  0.44686,	Adjusted R-squared:   0.4453 
F-statistic: 286.387 on 2 and 709 DF,  p-value: < 2.22e-16
\end{verbatim}
\end{kframe}
\end{knitrout}
    \end{frame}

    \begin{frame}[fragile]
      \fontsize{8}{8}
\begin{knitrout}
\definecolor{shadecolor}{rgb}{0.137, 0.137, 0.137}\begin{kframe}
\begin{alltt}
\hlstd{> }\hlstd{Random.numbers} \hlkwb{<-} \hlkwd{rnorm}\hlstd{(}\hlkwd{nrow}\hlstd{(my.sample))}
\hlstd{> }\hlstd{model2} \hlkwb{<-} \hlkwd{lm}\hlstd{(Graduation.rate} \hlopt{~} \hlstd{Average.combined.SAT} \hlopt{+} \hlstd{In.state.tuition}
\hlstd{+ }               \hlopt{+} \hlstd{Random.numbers,} \hlkwc{data}\hlstd{=my.sample)}
\hlstd{> }\hlkwd{summary}\hlstd{(model2)}
\end{alltt}
\begin{verbatim}

Call:
lm(formula = Graduation.rate ~ Average.combined.SAT + In.state.tuition + 
    Random.numbers, data = my.sample)

Residuals:
      Min        1Q    Median        3Q       Max 
-45.59477  -9.13473   0.06836   8.75583  43.74968 

Coefficients:
                         Estimate   Std. Error  t value Pr(>|t|)    
(Intercept)          -8.433559857  4.378188630 -1.92627 0.054471 .  
Average.combined.SAT  0.061244215  0.004896088 12.50881  < 2e-16 ***
In.state.tuition      0.001248531  0.000111177 11.23012  < 2e-16 ***
Random.numbers        0.277098090  0.537299499  0.51572 0.606208    
---
Signif. codes:  0 '***' 0.001 '**' 0.01 '*' 0.05 '.' 0.1 ' ' 1

Residual standard error: 13.7537 on 708 degrees of freedom
  (19 observations deleted due to missingness)
Multiple R-squared:  0.447068,	Adjusted R-squared:  0.444725 
F-statistic: 190.816 on 3 and 708 DF,  p-value: < 2.22e-16
\end{verbatim}
\end{kframe}
\end{knitrout}
    \end{frame}

    \begin{frame}{Which model is the best?}
      \begin{itemize}[<+->]
        \item In general, we want to select the model that is the most \alert{parsimonious}, that is, the model that has the best combination of being simple with a high $R^2$.
        \item This is easier said than done---using Adjusted $R^2$ is not enough. We'll come back to this next week!
      \end{itemize}
    \end{frame}
  \end{darkframes}

\end{document}

\documentclass{beamer}\usepackage[]{graphicx}\usepackage[]{color}
%% maxwidth is the original width if it is less than linewidth
%% otherwise use linewidth (to make sure the graphics do not exceed the margin)
\makeatletter
\def\maxwidth{ %
  \ifdim\Gin@nat@width>\linewidth
    \linewidth
  \else
    \Gin@nat@width
  \fi
}
\makeatother

\definecolor{fgcolor}{rgb}{1, 0.894, 0.769}
\newcommand{\hlnum}[1]{\textcolor[rgb]{0.824,0.412,0.118}{#1}}%
\newcommand{\hlstr}[1]{\textcolor[rgb]{1,0.894,0.71}{#1}}%
\newcommand{\hlcom}[1]{\textcolor[rgb]{0.824,0.706,0.549}{#1}}%
\newcommand{\hlopt}[1]{\textcolor[rgb]{1,0.894,0.769}{#1}}%
\newcommand{\hlstd}[1]{\textcolor[rgb]{1,0.894,0.769}{#1}}%
\newcommand{\hlkwa}[1]{\textcolor[rgb]{0.941,0.902,0.549}{#1}}%
\newcommand{\hlkwb}[1]{\textcolor[rgb]{0.804,0.776,0.451}{#1}}%
\newcommand{\hlkwc}[1]{\textcolor[rgb]{0.78,0.941,0.545}{#1}}%
\newcommand{\hlkwd}[1]{\textcolor[rgb]{1,0.78,0.769}{#1}}%
\let\hlipl\hlkwb

\usepackage{framed}
\makeatletter
\newenvironment{kframe}{%
 \def\at@end@of@kframe{}%
 \ifinner\ifhmode%
  \def\at@end@of@kframe{\end{minipage}}%
  \begin{minipage}{\columnwidth}%
 \fi\fi%
 \def\FrameCommand##1{\hskip\@totalleftmargin \hskip-\fboxsep
 \colorbox{shadecolor}{##1}\hskip-\fboxsep
     % There is no \\@totalrightmargin, so:
     \hskip-\linewidth \hskip-\@totalleftmargin \hskip\columnwidth}%
 \MakeFramed {\advance\hsize-\width
   \@totalleftmargin\z@ \linewidth\hsize
   \@setminipage}}%
 {\par\unskip\endMakeFramed%
 \at@end@of@kframe}
\makeatother

\definecolor{shadecolor}{rgb}{.97, .97, .97}
\definecolor{messagecolor}{rgb}{0, 0, 0}
\definecolor{warningcolor}{rgb}{1, 0, 1}
\definecolor{errorcolor}{rgb}{1, 0, 0}
\newenvironment{knitrout}{}{} % an empty environment to be redefined in TeX

\usepackage{alltt}
\usepackage{preview}
\usepackage{../371g-slides}
\title{Model evaluation: problems with $p$-values}
\subtitle{Lecture 20}
\author{STA 371G}
\IfFileExists{upquote.sty}{\usepackage{upquote}}{}
\begin{document}
  

  

  \frame{\maketitle}

  % Show outline at beginning of each section
  \AtBeginSection[]{
    \begin{frame}<beamer>
      \tableofcontents[currentsection]
    \end{frame}
  }

  %%%%%%% Slides start here %%%%%%%

  \begin{darkframes}
    \begin{frame}{A quick refresher on $p$-values}
      \begin{itemize}[<+->]
        \item To do a study, we select a sample from a population, and collect a statistic of interest in that sample (e.g., a mean or proportion)
        \item For example, in marketing, we might A/B test two potential web online store designs.
        \item We randomly show design A (the existing design) or B (the new design) to customers, and then track how much money they spend.
        \item A $t$-test would be used to test the null hypothesis that the average amount spent with design A is the same as the amount spent with design B.
        \item If we find $p<.05$ (and that the new design results in significantly higher revenue than the old design) then we redo our store with the new design.
      \end{itemize}
    \end{frame}

    \begin{frame}
      The hypotheses we are testing:
      \[
        H_0 : \mu_A=\mu_B \qquad\text{vs}\qquad
        H_A : \mu_A<\mu B
      \]
      And:
      \[
        \text{$p$-value} = P(\text{seeing a difference as large as our sample} \mid \text{$H_0$ is true})
      \]
      \pause
      So a small $p$-value means that null hypothesis $H_0$ is not a tenable assumption---we'll reject it and assume instead that the alternative hypothesis $H_A$ is true.
    \end{frame}

    \begin{frame}
      Remember: a $p$-value is a \alert{conditional probability} that represents how likely it is that we would see a difference as extreme as the results we saw in the sample, under the assumption that $H_0$ is true.

      \bigskip

      As a result, it is sensitive to:
      \begin{itemize}
        \item \alert{Sample size}: Seeing the same difference in a larger sample will tend to result in a smaller $p$-value.
        \item \alert{Sampling variation}: More variation in the population will tend to result in a larger $p$-value.
      \end{itemize}
    \end{frame}

    \section{The consequence of running too many hypothesis tests}

    \begin{frame}{The mystery data set}
      \begin{itemize}
        \item The ``mystery'' data set from Lecture 10 contained a $Y$ variable and 20 $X$ variables that were \alert{completely random}.
        \item We built a regression predicting $Y$ from the $X$'s, and some of the slope coefficients end up being statistically significant!
      \end{itemize}
    \end{frame}

    \begin{frame}[fragile]
      \fontvsm
\begin{knitrout}
\definecolor{shadecolor}{rgb}{0.137, 0.137, 0.137}\begin{kframe}
\begin{verbatim}
...
              Estimate Std. Error t value Pr(>|t|)   
(Intercept)  0.0048932  0.0315499   0.155  0.87678   
X1          -0.0242680  0.0311808  -0.778  0.43658   
X2           0.0094610  0.0325268   0.291  0.77121   
X3          -0.0164687  0.0316452  -0.520  0.60289   
X4           0.0332419  0.0315227   1.055  0.29190   
X5          -0.0003434  0.0320555  -0.011  0.99146   
X6          -0.0235806  0.0311449  -0.757  0.44916   
X7           0.0239592  0.0314237   0.762  0.44597   
X8          -0.0153691  0.0308775  -0.498  0.61878   
X9           0.0188475  0.0301545   0.625  0.53210   
X10          0.0816945  0.0310096   2.634  0.00856 **
X11          0.0300185  0.0314272   0.955  0.33972   
X12         -0.0189656  0.0307947  -0.616  0.53812   
X13         -0.0657305  0.0311022  -2.113  0.03482 * 
X14          0.0076949  0.0318495   0.242  0.80914   
X15          0.0227637  0.0316354   0.720  0.47197   
X16          0.0689893  0.0332165   2.077  0.03807 * 
X17         -0.0239150  0.0312032  -0.766  0.44361   
X18         -0.0002470  0.0316023  -0.008  0.99376   
X19         -0.0502899  0.0305908  -1.644  0.10051   
X20         -0.0628928  0.0325095  -1.935  0.05333 . 
...

\end{verbatim}
\end{kframe}
\end{knitrout}
    \end{frame}

    \begin{frame}
      \begin{itemize}[<+->]
        \item Suppose that the null hypothesis is always true.
        \item If you use $\alpha=.05$ as your cutoff for statistical significance, then you'll get a significant result 5\% of the time just by chance!
        \item If we have 20 predictor variables, we are running 20 hypothesis tests, and we'd \alert{expect} to see one of them be significant, just by chance.
        \item That means that when you are building a model from a large data set with many variables, you should expect to see some significant coefficients no matter what.
      \end{itemize}
    \end{frame}

    \section{Changing your analysis strategy based on a hypothesis test}

    \begin{frame}
      \begin{itemize}[<+->]
        \item Suppose you have a strong belief that there is a gender difference in propensity to buy a certain product.
        \item You build a logistic regression model and find that the gender coefficient is nonsignificant ($p>.05$).
        \item Maybe it would be significant for your older, 65+ customers?
          \begin{itemize} \item Nope---still nonsignificant. \end{itemize}
        \item Maybe it would be significant for your white customers?
          \begin{itemize} \item Nope---still nonsignificant. \end{itemize}
        \item Maybe it would be significant for customers in Texas?
          \begin{itemize} \item Nope---still nonsignificant. \end{itemize}
        \item Maybe it would be significant for customers in Oklahoma?
          \begin{itemize} \item Yes---$p=0.03$! \end{itemize}
        \item So it must be true that there is a significant gender difference in propensity to buy among customers in Oklahoma...
        \item ...or, you just got a false positive by running a lot of tests!
      \end{itemize}
    \end{frame}

    \begin{frame}{$p$-hacking}
      \begin{itemize}[<+->]
        \item This process is known as \alert{$p$-hacking}.
        \item In some sense, this is a natural part of the data analysis process---you don't necessarily know what effects are out there unless you look for them.
        \item But running too many tests can result in false positives!
      \end{itemize}
    \end{frame}

    \begin{frame}
      \fullpagepicture{xkcd}
    \end{frame}

    \begin{frame}
      \fullpagepicture{republicans}
    \end{frame}

    \begin{frame}
      \fullpagepicture{democrats}
    \end{frame}

    \begin{frame}
      \fullpagepicture{bem}
    \end{frame}

    \begin{frame}{What did the Bem study do?}
      \begin{itemize}[<+->]
        \item Reported on 9 different experiments of precognition (ESP)
        \item A typical example: participants see two images and pick one; the computer than randomly picks one of the images as the ``correct'' image, and the participant sees a ``positive'' image as a reward (or a ``negative'' image as punishment) if they happened to select the right one in advance.
        \item Participants picked the right image 51.7\% of the time ($t=2.39$, $p=.009$)
      \end{itemize}
    \end{frame}

    \begin{frame}{How did he prove that ESP exists?!}
      \begin{itemize}[<+->]
        \item Every time you do a hypothesis test, there is a 5\% chance that you'll get $p<.05$ just by chance!
        \item Bem reports on 9 experiments that came out with $p<.05$, but he doesn't report on all of the failed attempts! This is called the \alert{file-drawer problem}.
        \item Many researchers later tried to replicate Bem's results, and failed.
      \end{itemize}
    \end{frame}

    \section{Stopping an experiment based on a hypothesis test}

    \begin{frame}{Running A/B testing until we get significance}
      \begin{itemize}[<+->]
        \item Consider a similar situation: we run an A/B test where a customer is presented with two different offers.
        \item After each customer, we test the null hypothesis that the proportion of customers that prefer offer A is the same as the proportion that prefer offer B.
        \item We try this with up to 10,000 customers, but stop if we get a $p$-value less than .05.
      \end{itemize}
    \end{frame}

    \begin{frame}
      Let's simulate a situation where the customers in the population are just selecting randomly between offers A and B.

      \bigskip\pause

      In theory, the proportion of customers choosing offer A should stay near $0.5$ and we should get through all 10,000 customers with $p>.05$ each time.
    \end{frame}

    \begin{frame}
      The \textcolor{pink}{pink} line shows the $p$-value after each customer, and the \textcolor{cyan}{blue} line shows the proportion of customers that chose offer A.
\begin{knitrout}
\definecolor{shadecolor}{rgb}{0.137, 0.137, 0.137}
\input{/tmp/figures/unnamed-chunk-3-1.tex}

\end{knitrout}
    \end{frame}

    \begin{frame}
      \begin{itemize}
        \item Running the hypothesis test after each customer provides too many opportunities for false positives.
        \item Better: decide in advance how long you'll run the test.
      \end{itemize}
    \end{frame}

    \section{Using $p$-values responsibly}

    \begin{frame}{Two phases of analysis}
      \begin{itemize}
        \item Think of any analysis you do as having two phases, rather than one:
          \begin{itemize}
            \item The \alert{exploratory phase} where you try out different conditions, different subgroups, different measures of success, etc. until you find something that you \emph{think} works.
            \item The \alert{confirmatory phase} where you try to replicate your results in a new sample.
          \end{itemize}
      \end{itemize}
    \end{frame}

    \begin{frame}{The exploratory phase in regression analysis}
      \begin{itemize}
        \item Try looking at a large number of variables.
        \item Try looking at different subsets---e.g. only women, or only men; only large companies, or only small companies; etc.
        \item Use low $p$-values as a guide to what \emph{might} generalize to the larger population, but take them with a grain (cannister?!) of salt.
      \end{itemize}
    \end{frame}

    \begin{frame}{The confirmatory phase in regression analysis}
      \begin{itemize}
        \item Using what you think is the best model from the exploratory phase, see if your results generalize to a completely new sample.
        \item Now you can trust that your $p$-values are not misleading, since you are only running a single test on this sample!
      \end{itemize}
    \end{frame}

    \section{Survivorship bias}

    \begin{frame}
      \fullpagepicture{funds}
    \end{frame}

    \begin{frame}{Which funds beat the market?}
      \begin{itemize}[<+->]
        \item In any given year, the vast majority of managed funds trail the S\&P 500.
        \item Some managed funds do outperform the S\&P 500---but the high performers are \alert{different} every year.
        \item Why are there no funds that consistently outperform the market?
      \end{itemize}
    \end{frame}

    \begin{frame}{Survivorship bias}
      \begin{itemize}[<+->]
        \item The worst performers are the ones most likely to be shut down every year.
        \item So the situation is even worse than it looks!
        \item This phenomenon is called \alert{survivorship bias}: the survivors are not necessarily representative of the whole.
        \item Survivorship bias is why we should be careful about trying to emulate the characteristics of the most successful people or companies!
      \end{itemize}
    \end{frame}

    \section{Practical versus statistical significance}

    \begin{frame}
      \begin{itemize}[<+->]
        \item We show design A to 10,000 customers and design B to 10,000 customers.
        \item Design A results in a mean of \$40 spent, with an SD of \$20.
        \item Design B results in a mean of \$41 spent, with an SD of \$20.
        \item A one-sample $t$-test of the null hypothesis $\mu_A=\mu_B$ against $\mu_A<\mu_B$ gives a $p$-value of $0.0002$.
        \item So we should redesign the store with design B, right?
      \end{itemize}
    \end{frame}

    \begin{frame}{But:}
      \begin{itemize}[<+->]
        \item What if it would cost millions of dollars to do, and we would never recoup our investment?
        \item What if there is another design that would cost just as much to implement but would deliver a much higher ROI?
      \end{itemize}
    \end{frame}

    \begin{frame}{Practical vs statistical significance}
      \begin{itemize}[<+->]
        \item We tend to misuse $p$-values as a proxy for ``important,'' but $p$-values really just tell us whether our sample is consistent with the null hypothesis being true.
        \item A small $p$-value means that we can be confident that designs A and B do not result in exactly the same revenue per user---is the difference \alert{statistically significant}...
        \item ...but that's not answer to the question we really have: is the difference large enough for us to care about---is the difference \alert{practically significant}?
      \end{itemize}
    \end{frame}

    \begin{frame}
      \begin{itemize}[<+->]
        \item What ``practical'' means depends on your business situation: \alert{is the benefit worth the cost?}
        \item Design B might be statistically significantly better than design A, but the online store redesign might cost millions to execute.
        \item One fund might outperform the market in a statistically significant way, but the difference might be partially (or completely!) offset by the fund's fees.
      \end{itemize}
    \end{frame}
  \end{darkframes}
\end{document}

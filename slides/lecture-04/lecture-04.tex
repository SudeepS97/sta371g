\documentclass{beamer}\usepackage[]{graphicx}\usepackage[]{color}
%% maxwidth is the original width if it is less than linewidth
%% otherwise use linewidth (to make sure the graphics do not exceed the margin)
\makeatletter
\def\maxwidth{ %
  \ifdim\Gin@nat@width>\linewidth
    \linewidth
  \else
    \Gin@nat@width
  \fi
}
\makeatother

\definecolor{fgcolor}{rgb}{1, 0.894, 0.769}
\newcommand{\hlnum}[1]{\textcolor[rgb]{0.824,0.412,0.118}{#1}}%
\newcommand{\hlstr}[1]{\textcolor[rgb]{1,0.894,0.71}{#1}}%
\newcommand{\hlcom}[1]{\textcolor[rgb]{0.824,0.706,0.549}{#1}}%
\newcommand{\hlopt}[1]{\textcolor[rgb]{1,0.894,0.769}{#1}}%
\newcommand{\hlstd}[1]{\textcolor[rgb]{1,0.894,0.769}{#1}}%
\newcommand{\hlkwa}[1]{\textcolor[rgb]{0.941,0.902,0.549}{#1}}%
\newcommand{\hlkwb}[1]{\textcolor[rgb]{0.804,0.776,0.451}{#1}}%
\newcommand{\hlkwc}[1]{\textcolor[rgb]{0.78,0.941,0.545}{#1}}%
\newcommand{\hlkwd}[1]{\textcolor[rgb]{1,0.78,0.769}{#1}}%
\let\hlipl\hlkwb

\usepackage{framed}
\makeatletter
\newenvironment{kframe}{%
 \def\at@end@of@kframe{}%
 \ifinner\ifhmode%
  \def\at@end@of@kframe{\end{minipage}}%
  \begin{minipage}{\columnwidth}%
 \fi\fi%
 \def\FrameCommand##1{\hskip\@totalleftmargin \hskip-\fboxsep
 \colorbox{shadecolor}{##1}\hskip-\fboxsep
     % There is no \\@totalrightmargin, so:
     \hskip-\linewidth \hskip-\@totalleftmargin \hskip\columnwidth}%
 \MakeFramed {\advance\hsize-\width
   \@totalleftmargin\z@ \linewidth\hsize
   \@setminipage}}%
 {\par\unskip\endMakeFramed%
 \at@end@of@kframe}
\makeatother

\definecolor{shadecolor}{rgb}{.97, .97, .97}
\definecolor{messagecolor}{rgb}{0, 0, 0}
\definecolor{warningcolor}{rgb}{1, 0, 1}
\definecolor{errorcolor}{rgb}{1, 0, 0}
\newenvironment{knitrout}{}{} % an empty environment to be redefined in TeX

\usepackage{alltt}
\usepackage{../371g-slides}
\title{Inference for simple regression 2}
\subtitle{Lecture 4}
\author{STA 371G}
\IfFileExists{upquote.sty}{\usepackage{upquote}}{}
\begin{document}
  
  
  

  \frame{\maketitle}

  % Show outline at beginning of each section
  \AtBeginSection[]{
    \begin{frame}<beamer>
      \tableofcontents[currentsection]
    \end{frame}
  }

  %%%%%%% Slides start here %%%%%%%

  \begin{darkframes}
    \begin{frame}{Announcements and Logistics}
      \begin{itemize}
        \item HW 1 is now posted on Canvas/MyStatLab, and is due on Thursday.
        \item In addition to completing the assignment in MyStatLab, submit an R script file through Canvas that contains the commands you used to solve the problems.
      \end{itemize}
    \end{frame}

    \begin{frame}
      In finance, the $\beta$ of an asset indicates its volatility relative to the market. An asset with:
      \pause
      \begin{itemize}[<+->]
        \item $\beta=1$ rises and falls with the market as a whole.
        \item $\beta>1$ is \textbf{more} volatile than the market as a whole.
        \item $\beta<1$ is \textbf{less} volatile than the market as a whole.
      \end{itemize}
      \pause
      $\beta$ is just the slope of the regression line (i.e. $\hat\beta_1$) when we regress the asset's weekly returns against the weekly returns of a market index.
    \end{frame}

    \begin{frame}[fragile]{W5000 (Wilshire 5000, a broad market index)}
      \fontsize{10}{10}\selectfont
\begin{knitrout}
\definecolor{shadecolor}{rgb}{0.137, 0.137, 0.137}\begin{kframe}
\begin{alltt}
\hlstd{> }\hlkwd{hist}\hlstd{(stock.market}\hlopt{$}\hlstd{W5000,} \hlkwc{col}\hlstd{=}\hlstr{'green'}\hlstd{,}
\hlstd{+ }  \hlkwc{main}\hlstd{=}\hlstr{''}\hlstd{,} \hlkwc{xlab}\hlstd{=}\hlstr{'W5000 return as a % of previous week close'}\hlstd{)}
\end{alltt}
\end{kframe}
\input{/tmp/figures/unnamed-chunk-3-1.tex}

\end{knitrout}
      \note{Make connection to efficient market hypothesis}
    \end{frame}

    \begin{frame}[fragile]{W5000 (Wilshire 5000, a broad market index)}
      \fontsize{10}{10}\selectfont
\begin{knitrout}
\definecolor{shadecolor}{rgb}{0.137, 0.137, 0.137}\begin{kframe}
\begin{alltt}
\hlstd{> }\hlkwd{hist}\hlstd{(stock.market}\hlopt{$}\hlstd{W5000,} \hlkwc{col}\hlstd{=}\hlstr{'green'}\hlstd{,}
\hlstd{+ }  \hlkwc{main}\hlstd{=}\hlstr{''}\hlstd{,} \hlkwc{xlab}\hlstd{=}\hlstr{'W5000 return as a % of previous week close'}\hlstd{)}
\end{alltt}
\end{kframe}
\input{/tmp/figures/unnamed-chunk-4-1.tex}

\end{knitrout}
      \note{Make connection to efficient market hypothesis}
      \lc
    \end{frame}


    \begin{frame}[fragile]{Amazon (AMZN)}
\begin{knitrout}
\definecolor{shadecolor}{rgb}{0.137, 0.137, 0.137}\begin{kframe}
\begin{alltt}
\hlstd{> }\hlkwd{plot}\hlstd{(AMZN} \hlopt{~} \hlstd{W5000,} \hlkwc{data}\hlstd{=stock.market,}
\hlstd{+ }  \hlkwc{pch}\hlstd{=}\hlnum{16}\hlstd{,} \hlkwc{col}\hlstd{=}\hlstr{'cyan'}\hlstd{)}
\end{alltt}
\end{kframe}
\input{/tmp/figures/unnamed-chunk-5-1.tex}

\end{knitrout}
      \note{Interpret the meaning of a point (x,y) together}
      \lc
    \end{frame}

    \begin{frame}[fragile]
\begin{knitrout}
\definecolor{shadecolor}{rgb}{0.137, 0.137, 0.137}
\input{/tmp/figures/unnamed-chunk-6-1.tex}

\end{knitrout}

      The regression line is
      \[
        \widehat{\text{AMZN}} = 0.4 + 1.13 \cdot\text{W5000},
      \]
      with $R^2=0.22$ and $p=\ensuremath{7.8\times 10^{-16}}$.
      \lc
    \end{frame}

    \begin{frame}{Interpreting the regression statistics}
      \begin{itemize}[<+->]
        \item $\hat\beta_1=1.13$ (``$\beta$'') is the predicted increase in returns for AMZN when W5000 returns increase by 1 percentage point---since this is $>1$ AMZN will swing more than the market as a whole
        \item $R^2=0.22$ indicates how closely AMZN tracks W5000 (the market as a whole)
        \item $p=\ensuremath{7.8\times 10^{-16}}$ tells us whether we can reject the null hypothesis that AMZN does not move with the market at all \pause (we can! since $p$ is small)
      \end{itemize}
    \end{frame}

    \begin{frame}{Simple regression assumptions}
      We need four things to be true for statistical inference (i.e., hypothesis tests, $p$-values, confidence intervals) to work for regression:
      \pause
      \begin{enumerate}
        \item The errors are independent.
        \item $Y$ is a linear function of $X$ (except for the errors).
        \item The errors are normally distributed.
        \item The variance of $Y$ is the same for any value of $X$ (``homoscedasticity'').
      \end{enumerate}
    \end{frame}

    \begin{frame}{Assumption 1: Independence of errors}
      Independence means that knowing the error (over-/under-prediction by the regression line) for one case doesn't tell you anything about the error for another case.
    \end{frame}

    \begin{frame}{Assumption 1: Independence of errors}
      \begin{itemize}[<+->]
        \item From last time: Knowing how much more Bob drinks than expected (based on the age he started drinking) doesn't give us any suggestion as to how much more Lisa drinks than expected.
        \item From today: Knowing how much better or worse AMZN performs relative to the market \emph{last} week doesn't tell us anything about how much better or worse AMZN performs relative to the market \emph{this} week, if we believe the efficient market hypothesis.
        \item \alert{But:} Time-series data often violates the independence assumption!
        \item We can often only verify this assumption by thinking about the situation conceptually.
      \end{itemize}
    \end{frame}

    \begin{frame}{Simple regression assumptions}
      We need four things to be true for statistical inference (i.e., hypothesis tests, $p$-values, confidence intervals) to work for regression:
      \begin{enumerate}
        \item The errors are independent. \greencheckmark
        \item $Y$ is a linear function of $X$ (except for the errors).
        \item The errors are normally distributed.
        \item The variance of $Y$ is the same for any value of $X$ (``homoscedasticity'').
      \end{enumerate}
    \end{frame}

    \begin{frame}{Assumption 2: Linearity}
      Step 1: Visually examine to ensure a line is a good fit for the data:
\begin{knitrout}
\definecolor{shadecolor}{rgb}{0.137, 0.137, 0.137}
\input{/tmp/figures/unnamed-chunk-7-1.tex}

\end{knitrout}
    \end{frame}

    \begin{frame}{Assumption 2: Linearity}
      Each point has a \textbf{residual} ($Y-\hat Y$); this is the over/under-prediction of the model (red lines).
\begin{knitrout}
\definecolor{shadecolor}{rgb}{0.137, 0.137, 0.137}
\input{/tmp/figures/unnamed-chunk-8-1.tex}

\end{knitrout}
    \end{frame}

    \begin{frame}[fragile]{Assumption 2: Linearity}
      \fontsize{10}{10}\selectfont
      A \textbf{residual plot} (of residuals vs $X$) helps us ensure that there is not subtle nonlinearity. We want to see \textbf{no trend} in this plot:
\begin{knitrout}
\definecolor{shadecolor}{rgb}{0.137, 0.137, 0.137}\begin{kframe}
\begin{alltt}
\hlstd{> }\hlstd{model} \hlkwb{<-} \hlkwd{lm}\hlstd{(AMZN} \hlopt{~} \hlstd{W5000,} \hlkwc{data}\hlstd{=stock.market)}
\hlstd{> }\hlkwd{plot}\hlstd{(stock.market}\hlopt{$}\hlstd{W5000,} \hlkwd{resid}\hlstd{(model),}
\hlstd{+ }  \hlkwc{pch}\hlstd{=}\hlnum{16}\hlstd{,} \hlkwc{col}\hlstd{=}\hlstr{'green'}\hlstd{,} \hlkwc{xlab}\hlstd{=}\hlstr{'W5000'}\hlstd{,} \hlkwc{ylab}\hlstd{=}\hlstr{'Residuals'}\hlstd{)}
\end{alltt}
\end{kframe}
\input{/tmp/figures/unnamed-chunk-9-1.tex}

\end{knitrout}
    \end{frame}

    \begin{frame}{Simple regression assumptions}
      We need four things to be true for statistical inference (i.e., hypothesis tests, $p$-values, confidence intervals) to work for regression:
      \begin{enumerate}
        \item The errors are independent. \greencheckmark
        \item $Y$ is a linear function of $X$ (except for the errors). \greencheckmark
        \item The errors are normally distributed.
        \item The variance of $Y$ is the same for any value of $X$ (``homoscedasticity'').
      \end{enumerate}
    \end{frame}

    \begin{frame}[fragile]{Assumption 3: Errors are normally distributed}
      Step 1: Look at a histogram of the residuals and ensure they are approximately normally distributed:
\begin{knitrout}
\definecolor{shadecolor}{rgb}{0.137, 0.137, 0.137}\begin{kframe}
\begin{alltt}
\hlstd{> }\hlkwd{hist}\hlstd{(}\hlkwd{resid}\hlstd{(model),} \hlkwc{col}\hlstd{=}\hlstr{'darkred'}\hlstd{,}
\hlstd{+ }  \hlkwc{xlab}\hlstd{=}\hlstr{'Residuals'}\hlstd{,} \hlkwc{main}\hlstd{=}\hlstr{''}\hlstd{)}
\end{alltt}
\end{kframe}
\input{/tmp/figures/unnamed-chunk-10-1.tex}

\end{knitrout}
    \end{frame}

    \begin{frame}[fragile]{Assumption 3: Errors are normally distributed}
      Step 2: Look at a Q-Q plot of the residuals and look for an approximately straight line:
\begin{knitrout}
\definecolor{shadecolor}{rgb}{0.137, 0.137, 0.137}\begin{kframe}
\begin{alltt}
\hlstd{> }\hlkwd{qqnorm}\hlstd{(}\hlkwd{resid}\hlstd{(model),} \hlkwc{main}\hlstd{=}\hlstr{''}\hlstd{)}
\end{alltt}
\end{kframe}
\input{/tmp/figures/unnamed-chunk-11-1.tex}

\end{knitrout}
    \end{frame}

    \begin{frame}{Simple regression assumptions}
      We need four things to be true for statistical inference (i.e., hypothesis tests, $p$-values, confidence intervals) to work for regression:
      \begin{enumerate}
        \item The errors are independent. \greencheckmark
        \item $Y$ is a linear function of $X$ (except for the errors). \greencheckmark
        \item The errors are normally distributed. \greencheckmark
        \item The variance of $Y$ is the same for any value of $X$ (``homoscedasticity'').
      \end{enumerate}
    \end{frame}

    \begin{frame}[fragile]{Assumption 4: The variance of $Y$ is the same for any value of $X$}
      Look for the residual plot to have roughly equal vertical spread all the way across:
\begin{knitrout}
\definecolor{shadecolor}{rgb}{0.137, 0.137, 0.137}
\input{/tmp/figures/unnamed-chunk-12-1.tex}

\end{knitrout}
    \end{frame}

    \begin{frame}{Simple regression assumptions}
      We need four things to be true for statistical inference (i.e., hypothesis tests, $p$-values, confidence intervals) to work for regression:
      \begin{enumerate}
        \item The errors are independent. \greencheckmark
        \item $Y$ is a linear function of $X$ (except for the errors). \greencheckmark
        \item The errors are normally distributed. \greencheckmark
        \item The variance of $Y$ is the same for any value of $X$ (``homoscedasticity''). \greencheckmark
      \end{enumerate}
      \pause
      \alert{We always need to check these assumptions before interpreting $p$-values or confidence intervals!}
      \lc
    \end{frame}

    \begin{frame}{An example where an assumption fails}
      This is a data set of social worker salaries based on years of experience. Which assumption might be violated here?
\begin{knitrout}
\definecolor{shadecolor}{rgb}{0.137, 0.137, 0.137}
\input{/tmp/figures/unnamed-chunk-13-1.tex}

\end{knitrout}
    \end{frame}

    \begin{frame}{An example where an assumption fails}
\begin{knitrout}
\definecolor{shadecolor}{rgb}{0.137, 0.137, 0.137}
\input{/tmp/figures/unnamed-chunk-14-1.tex}

\end{knitrout}
      \note{Heteroscedasticity!}
    \end{frame}

  \end{darkframes}

\end{document}
